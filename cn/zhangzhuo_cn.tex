
%----------------------------------------------------------------------------------------
%	PACKAGES AND OTHER DOCUMENT CONFIGURATIONS
%----------------------------------------------------------------------------------------

\documentclass{resume} % Use the custom resume.cls style

\usepackage[left=0.75in,top=0.6in,right=0.75in,bottom=0.6in]{geometry} % Document margins

\name{张卓} % Your name
\address{18600162706 mycinbrin@gmail.com}
\address{清华大学紫荆公寓2\#609A} % Your address

\begin{document}
\vspace{-1.5em}
%----------------------------------------------------------------------------------------
%	EDUCATION SECTION
%----------------------------------------------------------------------------------------
\begin{rSection}{教育背景}
\begin{eSubsection}{清华大学计算机系}{北京}{计算机科学与技术专业}{08/2011 - 07/2015}
{经济学第二学位}{09/2012 - 06/2015}
\item \textbf{Major GPA: 88.0/100};
\item 获奖情况:
\begin{itemize}
\itemsep -0.5em \vspace{-0.5em}
\item[$\cdot$] 张明华综合奖学金 (20\%) 新生奖学金 (20\%)
\end{itemize}
\item 课程:
	\begin{itemize}
	\itemsep -0.5em \vspace{-0.5em}
	\item[$\cdot$] 人工智能,高性能计算,高级数据库,数值分析,计算机系统结构,网络编程,搜索引擎
	\end{itemize}
\end{eSubsection}
\vspace{-0.8em}
\begin{rSubsection}{香港大学计算机系}{香港}{校派交换学习}{08/2013 - 12/2013}
\item 课程:
	\begin{itemize}
	\itemsep -0.5em \vspace{-0.5em}
	\item[$\cdot$] 编译原理,操作系统,高级算法,模式识别,数据库
	\end{itemize}
\end{rSubsection}
\end{rSection}
\vspace{-1.0em}
%----------------------------------------------------------------------------------------
%	WORK EXPERIENCE SECTION
%----------------------------------------------------------------------------------------

\begin{rSection}{工作经验}
\begin{rSubsection}{Hulu}{北京}{推荐与个性化组}{02/2014 - 07/2014}
\vspace{-0.4em}
\item 构建一个供研究人员可视化评估算法的网站(Django)
\item 依据日志统计点击率等信息(Hadoop)
\item 实现了几个在线服务API(Java)
\end{rSubsection}
\vspace{-0.8em}
\begin{rSubsection}{Face++}{北京}{核心技术组}{12/2013 - 01/2014}
\vspace{-0.4em}
\item 实现相同图片过滤算法,利用图片分块后计算的色矩特征比较
\item 将图片过滤系统实现为Web服务,利用Redis队列协同前后台
\end{rSubsection}
\end{rSection}
\vspace{-1.0em}
\begin{rSection}{技能}
	\begin{tabular}{ @{} >{\bfseries}l @{\hspace{6ex}} l }
	熟练 & C++, C, Python, Bash, SQL, Linux常用工具 \\
	熟悉 & Javascript, Haskell, Java, Hadoop, Django, Git \\
	用过 & Scala, Node.js, Unix系统编程
	\end{tabular}
\end{rSection}
\begin{rSection}{项目经历}
\item {\bf 微博压力检测:} 新浪微博用户的心理压力检测.我是实现了算法后端,利用词袋模型做分类预测,利用TF-IDF做压力关键词提取.
\item {\bf Kademlia:} 利用Python Socket实现一个P2P用户分享文件和信息的网络软件.使用了Kademlia分布式哈希表.
\item {\bf 新加坡地图:} 使用Js Google Map API和Flask做出一个新加坡地图.主要为了展示几种高级字符串检索相关的算法的性能.使用Ctypes调用C++动态库.
\item {\bf MiniCompiler:} 利用flex和bison构建一个编译器,源语言类似于C,目标汇编语言类似于MIPS架构
\item {\bf AntiMalware:} 利用标注好的Android App恶意和普通数据集提取特征,之后用libweka中的机器学习算法分类
\end{rSection}
\end{document}
