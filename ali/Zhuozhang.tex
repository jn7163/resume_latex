
%----------------------------------------------------------------------------------------
%	PACKAGES AND OTHER DOCUMENT CONFIGURATIONS
%----------------------------------------------------------------------------------------

\documentclass{resume} % Use the custom resume.cls style

\usepackage[left=0.75in,top=0.6in,right=0.75in,bottom=0.6in]{geometry} % Document margins

\name{张卓} % Your name
\address{清华大学紫荆公寓2\#609A} % Your address
\address{18600162706}
\address{mycinbrin@gmail.com} % Your phone number and email

\begin{document}
\vspace{-1.5em}
%----------------------------------------------------------------------------------------
%	EDUCATION SECTION
%----------------------------------------------------------------------------------------
\begin{rSection}{教育背景}
\begin{eSubsection}{清华大学计算机系}{北京}{计算机科学与技术专业}{08/2011 - 07/2015}
{经济学第二学位}{09/2012 - 06/2015}
\item \textbf{Major GPA: 88.0/100};
\item 获奖情况:
\begin{itemize}
\itemsep -0.5em \vspace{-0.5em}
\item[$\cdot$] 张明华综合奖学金 (20\%)
\item[$\cdot$] 新生奖学金 (20\%)
\end{itemize}
\end{eSubsection}
\vspace{-0.8em}
\begin{rSubsection}{香港大学计算机系}{香港}{校派交换学习}{08/2013 - 12/2013}
\end{rSubsection}
\end{rSection}
\vspace{-1.0em}
%----------------------------------------------------------------------------------------
%	WORK EXPERIENCE SECTION
%----------------------------------------------------------------------------------------

\begin{rSection}{工作经验}
\begin{rSubsection}{Hulu}{北京}{推荐与个性化组}{02/2014 - 至今}
\vspace{-0.4em}
\item 在Hadoop平台上编写一些推荐算法,例如协同过滤
\item 负责推荐模块管理系统的维护与新功能添加
\item 设计实现基于剧集元信息的推荐模型
\end{rSubsection}
\vspace{-0.8em}
\begin{rSubsection}{Face++}{北京}{核心技术组}{12/2013 - 01/2014}
\vspace{-0.4em}
\item 实现相同图片过滤算法,利用图片分块后计算的色矩特征比较
\item 将图片过滤系统实现为Web服务,利用Redis队列协同前后台
\end{rSubsection}
\end{rSection}
\vspace{-1.0em}
\begin{rSection}{项目经历}
\begin{sSubsection}{Minicompiler}{10/2013}
\item 利用flex和bison构建一个编译器
\item 源语言类似于C,目标汇编语言类似于MIPS架构
\end{sSubsection}
\vspace{-0.8em}
\begin{sSubsection}{AntiMalware}{10/2013}
\item 获取Android恶意软件数据集,并提取静态特征500多个
\item 利用libweka对数据分析,并评估不同的分类算法
\item 开发一个简单的命令行工具,可判断新的App是否为恶意软件
\end{sSubsection}
\vspace{-0.8em}
\end{rSection}
\begin{rSection}{技能}
\begin{tabular}{ @{} >{\bfseries}l @{\hspace{6ex}} l }
熟练 & C, C++, Python, Bash, SQL, Linux常用工具 \\
熟悉 & Haskell, Java, Javascript, Hadoop, Django, Unix Programming, Git \\
用过 & Rails, Node.js, MPI
\end{tabular}
\end{rSection}
\end{document}
